\documentclass[conference]{IEEEtran}
\IEEEoverridecommandlockouts
% The preceding line is only needed to identify funding in the first footnote. If that is unneeded, please comment it out.
\usepackage{cite}
\usepackage{amsmath,amssymb,amsfonts}
\usepackage{algorithmic}
\usepackage{graphicx}
\usepackage{textcomp}
\usepackage{xcolor}
\usepackage[acronym]{glossaries}
\usepackage[english]{babel}
\usepackage[autolanguage]{numprint}
\usepackage{todonotes}
\usepackage{endnotes}
\usepackage{xcolor}
\usepackage{xspace}
\usepackage{numprint}


\renewcommand{\notesname}{Referenced URLs}
\newcommand{\citeurl}[1]{\endnote{{\scriptsize \url{#1}}}}

\newcommand{\toolname}[1]{{\scriptsize \texttt{#1}}}

\begin{document}

\title{Visualizing Git Repositories for Assessment in Software Engineering Education}
% \thanks{Identify applicable funding agency here. If none, delete this.}


\author{\IEEEauthorblockN{1\textsuperscript{st} Given Name Surname}
\IEEEauthorblockA{\textit{dept. name of organization (of Aff.)} \\
\textit{name of organization (of Aff.)}\\
City, Country \\
email address or ORCID}
\and
\IEEEauthorblockN{2\textsuperscript{nd} Given Name Surname}
\IEEEauthorblockA{\textit{dept. name of organization (of Aff.)} \\
\textit{name of organization (of Aff.)}\\
City, Country \\
email address or ORCID}
\and
\IEEEauthorblockN{3\textsuperscript{rd} Given Name Surname}
\IEEEauthorblockA{\textit{dept. name of organization (of Aff.)} \\
\textit{name of organization (of Aff.)}\\
City, Country \\
email address or ORCID}
\and
\IEEEauthorblockN{4\textsuperscript{th} Given Name Surname}
\IEEEauthorblockA{\textit{dept. name of organization (of Aff.)} \\
\textit{name of organization (of Aff.)}\\
City, Country \\
email address or ORCID}
\and
\IEEEauthorblockN{5\textsuperscript{th} Given Name Surname}
\IEEEauthorblockA{\textit{dept. name of organization (of Aff.)} \\
\textit{name of organization (of Aff.)}\\
City, Country \\
email address or ORCID}
\and
\IEEEauthorblockN{6\textsuperscript{th} Given Name Surname}
\IEEEauthorblockA{\textit{dept. name of organization (of Aff.)} \\
\textit{name of organization (of Aff.)}\\
City, Country \\
email address or ORCID}
}

\maketitle

\begin{abstract}
Practitioners apply various readily available tools to visualize version control system data (software repositories).
Since such tools target usually software engineering professionals but not educators, their potential benefits for assessment in software engineering education are not explored much.
In this paper we describe usage patterns of software repository visualizations, illustrate them on the tool prototype \toolname{git-truck}, 
% discuss challenges and opportunities of applying repository visualization for assessment in software engineering education.
and we discuss challenges for future tool builders.
\end{abstract}

\begin{IEEEkeywords}
\end{IEEEkeywords}

\section{Introduction}

\section{Related Work}

\section{Assumptions}

\section{The Tool Prototype: \toolname{git-truck}}

% The tool used in this case study is git-truck, v. 1.1.0. 
The tool \toolname{git-truck} is released under an open source license (MIT) and is available on Github\citeurl{https://github.com/git-truck/git-truck} and as Node package on the \toolname{npm} registry\citeurl{https://www.npmjs.com/package/git-truck}. 
It can be installed easily on any on which system \toolname{NodeJS} and the corresponding package manager \toolname{npm} are setup\citeurl{https://docs.npmjs.com/cli/v7/configuring-npm/install}.

Unlike many other software-as-a-service tools, \toolname{git-truck} is meant to be executed directly on personal computers.
That allows educators to assess also private or institutional repositories, which are usually not shared publicly via software forges like Github, Gitlab, Bitbucket, etc.

To assess a local \toolname{git} repository, the tool can be executed directly in a directory containing a repository via the command \texttt{npx run git-truck@latest}.
In case the tool is not installed yet, the previous command will download and install it automatically.

After execution, \toolname{git-truck} visualizes the file structure of a repository using hierarchical metric-enhanced layouts, such as, \emph{circle packing visualizations}~\cite{wang2006visualization} or tree maps~\cite{johnson1992treeviz}.
Both represent containment structures of directories and files, where the visual size of files is proportional to their size in bytes.~\todo{Really?}

\toolname{git-truck} supports interactive author-unification, i.e., it allows to group multiple authors into single \emph{logical} authors.
This is often required in education, since students often commit using multiple user names, when their \toolname{git} configuration differs across computers, such as, school terminals, private laptops, and home computers.
\toolname{git-truck}'s author-unification could rely completely on \toolname{git} \emph{mailmap} files\citeurl{https://git-scm.com/docs/gitmailmap}, but students configure them rarely correctly.
Additionally, \toolname{gittruck} supports co-author attribution.
That is, commit co-authors that are identified via the \texttt{Co-authored-by} tag in commit messages are extracted.
Such a feature is especially relevant, in times of a global COVID-19 pandemic in which students often work collaboratively via shared editors.

\todo{Finish with the notes from below}
% * It can highlight various evolutionary and structural properties on every file
%    * Number of commits per file, and aggregated to folders
%    * Main contributor - the developer with more than 50% of the lines changed in a file
%    * Single contributor - highlights those files which have been only touched by a single developer
%    * Programming languages - 
%    * …
% * It is highly interactive supporting filtering, zooming, details on demand
% * Is designed for ease of use and installation
% * <other stuff that’s relevant> 


% Finally, git-truck supports opening a folder of git projects as shown in the image below; this is a very trivial-looking feature, but it increases usability to an educator who needs to evaluate a large number of projects. 


Note, as the name of the tool suggests, it is only meant to visualize \emph{\toolname{git}} repositories.
Support of other version control systems that are used in software engineering education, such as, Mercurial, Fossil, etc. remains future work.







\section{Usage Patterns}

\subsection{Visualize Share of Work with a Truck Factor of One}

\subsection{Visualize Top Contributor to Detect Responsibility}

\subsection{Visualize Structural Properties to Estimate Software Architecture}

\subsection{Visualize Change Frequencies to Understand Work Effort Distribution}


\section{Discussion}

\subsection{Threats to Validity}



\bibliographystyle{IEEEtran}
\bibliography{bibliography}

\end{document}
